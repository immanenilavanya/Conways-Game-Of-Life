\documentclass{beamer}
\usepackage{graphicx}
\usepackage{hyperref}

\graphicspath{ {./images/} }
\mode<presentation>
{
\usetheme{Antibes}
\usecolortheme{dolphin}
\usefonttheme{default}
\setbeamertemplate{navigation symbols}{}
\setbeamertemplate{caption}[numbered]
}
\usepackage[english]{babel}
\section{Team Introduction}
\title{ Conway's game of life}
\author[Svecw]{I.Lakshmi Lavanya : 21B01A0562 : cse \\ K. Bharathi :\;\;\;\;\;\;\; \;\;\;21B01A0565 : cse \\ G. loshmi : \;\;\;\;\;\;\;\;\;\;\;\;\;\;21B01A0419 : ece \\ G.Jyothi sri :\;\;\;\;\;\;\;\;\;\;\; 21B01A0426 : ece }
\date{January 28 2023}

\begin{document}


\begin{frame}
   \titlepage
\end{frame}

\section{Problem Statement}
\begin{frame}
{\huge Introduction}
\begin{itemize}
    \item Conway's game of Life is played on an infinite two-dimensional rectangular grid of cells. 

     \item These simple rules are as follows: \\
* The cell stays alive if it has either 2 or 3 live neighbors

* The cell springs to life only when it has 3 live neighbors

\end{itemize}
\end{frame}

\section{}

\begin{frame}{\huge Approach}

\begin{itemize}
     \item Divide and Conquer
     \item Using rules as the base
     \item Identifies the control flow to call the functions
\end{itemize}

\vskip 1cm

\end{frame}

\section{New Packages}

\begin{frame}{\huge Learnings}

\begin{itemize}
\item\huge Pygame 
\begin{itemize}
\vskip 1cm
\item pygame.display
\item pygame.clock
\item pygame.surface
\item pygame.draw
\item pygame.mouse
\item pygame.events
\end {itemize}
\end{itemize}
\end{frame}
\section{}
\begin{frame}
{\huge Challenges}
\begin{itemize}
\item Choosing the right co-ordinates for elements in  grid\\ { \;-Used trail and error method}
\item Identifying the well suited built-in function \\ {\;-Cross-checking the requirements and built-in function properties}
\item Usage of Built-In functions \\ {\;-Acquired proper knowledge on them}
\end{itemize}
\end{frame}

\section{}

\begin{frame}{\huge Statistics}
\begin{itemize}
\item Number of lines : 106 lines
\item Number of function : 6 User defined functions
\begin{itemize}
\item cell\_layout()
\item layout()
\item commands()
\item neighbours()
\item ResetGrid()
\end{itemize}
\end {itemize}
\end{frame}
\begin{frame}{\huge Demo}
\begin{figure}[C:\Users\Adithya\Desktop]
    \centering
    \includegraphics[width=6cm]{output.png}
    \caption{Demo output}
    \label{fig:demo1}
\end{figure}
\end{frame}

\subsection{}
\begin{frame}\centering{\huge Thank you!}
\end{frame}
\end{document}
